\documentclass[10pt,a4paper]{article}
\usepackage[utf8]{inputenc}
\usepackage[german]{babel}
\usepackage[T1]{fontenc}
\usepackage{amsmath}
\usepackage{amsfonts}
\usepackage{amssymb}
\usepackage{graphicx}
\usepackage{lmodern}
\usepackage{fourier}
\usepackage[left=1cm,right=1cm,top=1cm,bottom=1cm]{geometry}
\begin{document}
\resizebox{\textwidth}{!}{
\begin{tabular}{l l l}
\textbf{Variable} & \textbf{Definition} & \textbf{Kapitel}\\
\hline
$\overline{x}$ & Arithmetisches Mittel &(6.1) \\
$\overline{x}_w$ & Gewichteter Mittelwert &(6.2) \\
$\overline{x}_G$ & Geometrisches Mittel &(6.4)\\
$\overline{x}_H$ & Harmonisches Mittel &(6.5)\\
$\overline{x}_{\alpha}$ & Getrimmtes &(6.7)\\
$s_x^2, \tilde{s}_x^2$ & Stichprobenvarianz &(6.20)\\
$s_x$ & Standardabweichung &(6.21)\\
$\overline{x}_1, \tilde{s}_{x1}^2$ & Schichtmittelwerte und Schichtvarianzen &(6.23)\\
$\sigma(X)$ & Standardabweichung ZV &(6.27)\\
$g_p, g_m$ & Quantilskoeffizient und Momentenkoeffizient &(6.30)\\
$s_{xy}$ & Kovarianz &(10.1)\\
$r_{xy}$ & Bravais-Pearson &(10.2)\\
$Cov(X,Y)$ & Kovarianz ZV &(10.4)\\
$\rho(X,Y)$ & Korrelation ZV &(10.5)\\
$r_{xy}^{SP}$ & Spearman &(10.8)
\end{tabular}
}
\end{document}